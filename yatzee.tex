\documentclass[a4paper,12pt]{article}
%For images
\usepackage{graphicx}
 
\addtolength{\oddsidemargin}{-.875in}
\addtolength{\evensidemargin}{-.875in}
\addtolength{\textwidth}{1.75in}
 
\addtolength{\topmargin}{-.875in}
\addtolength{\textheight}{1.75in}
 
\begin{document}

Yatzee has a few rules
You throw 6 dice and then pick out a subset of those dice and can do that
2 times. Total dice throws is therefore 3.

\paragraph{}
One can after any dice throw then decide from a set of rules to try and maximize
ones points. Examples could be 3 dice who have the same value, or full house. I will
now preceed to make an algorithm that can closely determine the optimal way of playing
Yatzee

\paragraph{}
Let's begin with a greedy algorithm. It will not reroll any dice and pick whatever gives
the highest reward in the near term. For rulepoint i, point function q and dice X,
for each round of the game

$$\max_{i\in I_t} q_i({X_1,...,X_6}), I_{t+1} = I_t - i$$




\section{Irreducability}

My intuition tells me that the game of Yatzee is irreducable, and adding
or subtracting rules would mean a completely different solution. i.e, there is
no general closed solution to a yatzee game, but instead has to either be analyzed
one by one or use maybe some form of reinforcment learning algorithm that
approximates the outcomes.

\section{Closed solution to a basic yatzee game}

I will begin with this, but I doubt it's generalizable. The only part
of the ruleset allowed for now will be the number of dice times their number
if they all have the same number.

$$S_t = c*$$

\end{document}